\section{Conclusion}
\label{sec.conclusion}

This work presented the principles behind the generation of an holographic and stereo visualization. The first is still not consolidated because it needs special hardware and high computational power; but it has the benefit of reproducing the light in all directions, from all viewpoints. The latter is a visualization technique already present in the stereo monitors, TVs and movie screens; but with a limited immersion due the the fixed viewpoint.

The new projection transformation for stereo displays using the OpenGL framework creates the illusion of an hologram. The human stereo vision requires only two viewpoints of the hologram, it was emulated in the stereo displays. The viewpoint is adjusted in real-time with the tracked position of the viewer.

The experiment shows a computer generated projection in a fishtank setup composed by two stereo displays. The hologram is emulated using negative stereo parallax and high precision tracking of the viewer. The holographic emulation is an effort to use the current technology to produce the same immersion of visualization of an hologram. New development in holographic technology will allow holographic displays, but will take at least half decade in order to have commercial products with this technology. The cost of the high end hardware necessary to create CGH are not yet proportional to the needs of this visualization technique, and this work shows that holographic immersive environment can be done today with out-of-the-shelf hardware. The  stereo displays and Kinect sensor are available for a fraction of the cost of a special hardware.

