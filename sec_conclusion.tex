\section{Conclusion}
\label{sec.conclusion}

This work presented the principles behind the generation of an holographic and stereo visualization. The first is still not consolidated because it needs special hardware and high computational power; but it has the benefit of surround viewing. The latter is a visualization technique already present in the stereo monitors, TVs and movie screens; but with a limited immersion due the the fixed viewpoint.

A new pipeline for stereo displays has been proposed to create the illusion of an hologram. The human stereo vision requires only two viewpoints of the hologram, it was emulated in the stereo displays. The viewpoint is adjusted in real-time with the tracked position of the viewer.

The results of the implementation show a computer generated projection in a FTRV environment composed by two stereo displays. The hologram is emulated using negative stereo parallax and high precision tracking of the viewer. The holographic emulation is an effort to use the current technology to produce the same immersion of visualization of an hologram. 

New development in holographic technology will allow holographic displays in the future~\cite{Lucente2012}, but how long will take to have commercial products with this technology is still open to discussion. This work shows that an immersive holographic environment can be done today with off-shelf hardware and new applications with holographic visualization can be developed with off-the-shelf hardware for a fraction of the cost of CGH.

