\section{Introduction}
\label{sec.introduction}

The holographic display is a new generation of 3D display technology pursued by many researchers~\cite{Lucente1992, Watlington1995, Lucente2012} and companies~\cite{schwerdtner2013} over the past decades. The holographic visualization, the incidence of a source beam of light on a fringe pattern, is a process that produces full-depth-cue 3D images at beyond eye resolution, floating in space, and with an extended color gamut. This holographic effect is simulated by the computer using Computer Generated Holography (CGH), and it has been labeled as the ultimate display technology~\cite{Slinger2005}. 

However, many CGH-based displays have a computational cost for projecting pixels that can far exceed other display types. The necessary amount of data for real-time rendering of the fringe pattern is not feasible using today's technology. There is a gap in the currently established 3D display technology and processing power for CGH at eye resolution. For example, a 60-cm-diagonal hologram has over $10^{12}$ samples --– the equivalent of a terabyte~\cite{Lucente2012}.  

Over the last decade, promising technologies have been proposed to decimate the amount of data~\cite{Lucente1992, schwerdtner2013} and allow 3D recording and reproduction by CGH. Real-time CGH displays are not yet consolidated because they need special hardware and high computational power~\cite{Lucente1992, Watlington1995, Lucente2012}. The additional computational operations add to the cost of such systems, particularly high-frame-rate interactive systems, and lower-cost, simpler display technologies are more appropriate for many applications without the need of immersion of a CGH. 

This work proposes a new stereo visualization technique to generate the immersion of a CGH without specialized hardware. The stereo visualization technique already present in the stereo monitors, TVs and theater screens have a limited immersion and a fixed viewpoint. If the content is computer generated, the tracking of the viewer can reproduce simultaneous viewpoints~\cite{Frohlich2005, Harris2010} and an higher degree of immersion is achieved. The viewing parameters are combined with stereo visualization and depth cues into a virtual screen, floating in the space, where an holographic projection is emulated. 

Another contribution of this paper is to determine an holographic visualization pipeline using an existing 3D visualization standard. The implementation of the holographic visualization uses standard viewing and motion software toolkits~\cite{schreiner2004, Burns2004, Zhang2012}, that have been emerged to supply the lack of standardization, and would benefit end users by delivering consistent and comprehensive 3D interaction across applications for user interface features like stereo and motion capture.

The paper is organized as follows: Section~\ref{sec.stereo} provides a briefly introduction in the stereo concepts, followed by a short survey in the existing stereo technology; Section~\ref{sec.visualization_pipeline} reviews the 3D visualization pipeline and the computation of the stereo pairs. The Section~\ref{sec.holographic_visualization} proposes an alternative to CGH using stereo pairs and an accompanying visualization pipeline. Results of implementation are shown in Section~\ref{sec.results} and the conclusions presented in Section~\ref{sec.conclusion}.


