\section{Introduction}
\label{sec.introduction}

This work proposes a new stereo visualization technique to generate the immersion of an holographic display without specialized hardware. Holographic displays are not consolidated because they need special hardware and high computational power~\cite{Lucente1992, Watlington1995, Lucente2012}; but it has the benefit of reproducing the light in all directions, from all viewpoints. The stereo visualization technique already present in the stereo monitors, TVs and movie screens have a limited immersion and a fixed viewpoint. If the content is computer generated, the tracking of the viewer can reproduce the viewpoint, but has limitations on the number of simultaneous viewers~\cite{Harris2010}.

Another contribution of this paper is an holographic transform using the existing OpenGL pipeline. Closely tied in to viewing parameters is 3D interaction, OpenGL standard has emerged to supply the lack of standardization, the need of higher development and support efforts; as well as stifling the proliferation of user interface features like stereo. A standard viewing software toolkit, along with a standard motion toolkit, would benefit end users by delivering consistent and comprehensive 3D interaction across applications.

The section is organized as follows: Section~\ref{sec.holography} provides and introduction in hologram generation; then Stereo is briefly discussed in Section~\ref{sec.stereographic_displays}; and followed by Section~\ref{sec.projection_review}, that reviews the OpenGL 3D projection. The following Sections~\ref{sec.hologram_emulation} and~\ref{sec.holographic_projection} propose an alternative to computer generated hologram and a new projection transform for hologram emulation, respectively, using stereo view pairs.

