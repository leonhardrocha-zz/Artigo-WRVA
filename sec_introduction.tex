\section{Introduction}
\label{sec.introduction}

Just recently, the recording and reproduction of a 3D image can be made at some level by a computer using Computer Generated Holography (CGH)~\cite{Lucente1992}. This process, has an unique ability to produce full-depth-cue 3D images at beyond eye resolution, floating in space, and with an extended color gamut has led some to label CGH the ultimate display technology~\cite{Slinger2005}. 

However, many CGH-based displays have a computational cost for projecting pixels that can far exceed other display types. Additional computational operations add to the cost of such systems, particularly high-frame-rate interactive systems. Real-time CGH displays have been proposed~\cite{Lucente1992, Watlington1995, Lucente2012}, but are not consolidated because they need special hardware and high computational power. When compared with other visualization techniques, CGH has the benefit of reproducing the light in all directions, from all viewpoints; but for many applications, lower-cost, simpler display technologies will be more appropriate. 

This work proposes a new stereo visualization technique to generate the immersion of an CGH without specialized hardware.  The stereo visualization technique already present in the stereo monitors, TVs and movie screens have a limited immersion and a fixed viewpoint. If the content is computer generated, the tracking of the viewer can reproduce the viewpoint, but has limitations on the number of simultaneous viewers~\cite{Harris2010}.

Another contribution of this paper is an holographic transform using the existing OpenGL pipeline~\cite{schreiner2004}. Closely tied in to viewing parameters is 3D interaction, OpenGL standard has emerged to supply the lack of standardization, the need of higher development and support efforts; as well as stifling the proliferation of user interface features like stereo. A standard viewing software toolkit, along with a standard motion toolkit, would benefit end users by delivering consistent and comprehensive 3D interaction across applications.

The paper is organized as follows: Section~\ref{sec.stereo} provides a briefly introduction in the stereo concepts, followed by a short survey in the existing stereo technology; Section~\ref{sec.visualization_pipeline} reviews the OpenGL 3D projection and shows the computation of the stereo pairs. The Section~\ref{sec.holographic_visualization} proposes an alternative to CGH using stereo pairs and an accompanying visualization pipeline. Results of implementation are shown in Section~\ref{sec.results} and the conclusions presented in Section~\ref{sec.conclusion}.


